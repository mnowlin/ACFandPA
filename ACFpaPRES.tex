\documentclass{beamer}
%\usepackage{beamerthemeHannover}
\usepackage[round]{natbib}
\usepackage{amsmath}
\setbeamercovered{invisible}
\usepackage[mathletters]{ucs}
\usepackage[utf8x]{inputenc}
%% \setlength{\parindent}{0pt}
%% \setlength{\parskip}{6pt plus 2pt minus 1pt}
%% \setcounter{secnumdepth}{0}

\title{Coalitions and Principal-Agency Dynamics: Delegation and the Advocacy
Coalition Framework}
\author{Matthew C. Nowlin} 
\institute
{Department of Political Science \\
College of Charleston \\
\url{nowlinmc@cofc.edu}}
\date{October 2014}

% remove navigation bar
\setbeamertemplate{navigation symbols}{}

% transparent overlays
% \setbeamercovered{transparent}

% for natbib
\def\newblock{}

\begin{document}
%\begin{frame}{}
  \titlepage

\begin{frame}{Research Question(s)}

\begin{itemize}
\itemsep1pt\parskip0pt\parsep0pt
\item
  How do advocacy coalitions impact the dynamics of principal-agent
  relationships?
\end{itemize}

\vspace{0.25in}

\begin{itemize}
\itemsep1pt\parskip0pt\parsep0pt
\item
  What are the implications of principals and agents within the same
  (different) advocacy coalition?
\end{itemize}

\end{frame}

\begin{frame}{Delegation and Control}

\begin{itemize}
\itemsep1pt\parskip0pt\parsep0pt
\item
  Delegation

  \begin{itemize}
  \itemsep1pt\parskip0pt\parsep0pt
  \item
    Policy Uncertainty
  \item
    Ally Principle
  \end{itemize}
\end{itemize}

\vspace{0.25in}

\begin{itemize}
\itemsep1pt\parskip0pt\parsep0pt
\item
  Control

  \begin{itemize}
  \itemsep1pt\parskip0pt\parsep0pt
  \item
    \emph{ex ante} / \emph{ex post}
  \item
    Political Uncertainty
  \end{itemize}
\end{itemize}

\vspace{0.25in}

\begin{itemize}
\itemsep1pt\parskip0pt\parsep0pt
\item
  Institutional Dominance
\end{itemize}

\end{frame}

\begin{frame}{Bureaucratic Motivations}

\begin{itemize}
\itemsep1pt\parskip0pt\parsep0pt
\item
  Neutral Competence

  \begin{itemize}
  \itemsep1pt\parskip0pt\parsep0pt
  \item
    Merit System
  \item
    \emph{Objective Technician}
  \end{itemize}
\item
  Responsive Agents

  \begin{itemize}
  \itemsep1pt\parskip0pt\parsep0pt
  \item
    Agency Theory
  \item
    \emph{Client Advocates}
  \end{itemize}
\item
  Self-Interest

  \begin{itemize}
  \itemsep1pt\parskip0pt\parsep0pt
  \item
    \emph{Climbers} / \emph{Conservers}
  \item
    \emph{Budget Maximizers}
  \end{itemize}
\item
  \textbf{Policy}

  \begin{itemize}
  \itemsep1pt\parskip0pt\parsep0pt
  \item
    Driven by beliefs
  \item
    \emph{Issue Advocates}
  \item
    \emph{Zealots}
  \end{itemize}
\end{itemize}

\end{frame}

\begin{frame}{Advocacy Coalition Framework}

\begin{itemize}
\itemsep1pt\parskip0pt\parsep0pt
\item
  Policy actors within subsystems aggregate into coalitions based on
  shared policy beliefs and coordinate to achieve shared goals
\end{itemize}

\vspace{0.25in}

\begin{itemize}
\itemsep1pt\parskip0pt\parsep0pt
\item
  Policy-motivated and belief driven
\end{itemize}

\end{frame}

\begin{frame}{Coalitions and Delegation/Control}

\begin{itemize}
\itemsep1pt\parskip0pt\parsep0pt
\item
  Assumptions

  \begin{itemize}
  \itemsep1pt\parskip0pt\parsep0pt
  \item
    Elected officials (principals) and bureaucrats (agents) are members
    of advocacy coalitions \pause
    \vspace{0.1in}
  \item
    Principals are present across multiple policymaking institutions in
    which agents are to be responsive \pause
    \vspace{0.1in}
  \item
    Bureaucratic motivations are a function of subsystem type; unitary,
    collaborative, or adversarial \pause
    \vspace{0.1in}
  \item
    Delegation and control mechanisms are embedded within policy designs
    that are often a result of conflict and compromise
  \end{itemize}
\end{itemize}

\end{frame}

\begin{frame}{Used Nuclear Fuel Management}

\begin{itemize}
\itemsep1pt\parskip0pt\parsep0pt
\item
  Adversarial subsystem
\end{itemize}

\vspace{0.1in}

\begin{itemize}
\itemsep1pt\parskip0pt\parsep0pt
\item
  Two coalitions

  \begin{itemize}
  \itemsep1pt\parskip0pt\parsep0pt
  \item
    Pro-Yucca Mountain: Congress, industry groups, and the DOE
  \item
    Anti-Yucca Mountain: the state of Nevada and environmental groups
  \end{itemize}
\end{itemize}

\vspace{0.1in}

\begin{itemize}
\itemsep1pt\parskip0pt\parsep0pt
\item
  Increasing conflict lead to increased control of the DOE
\end{itemize}

\vspace{0.1in}

\begin{itemize}
\itemsep1pt\parskip0pt\parsep0pt
\item
  Budget
\end{itemize}

\end{frame}

\begin{frame}{Discussion}

\begin{itemize}
\itemsep1pt\parskip0pt\parsep0pt
\item
  Elected officials in subsystems
\end{itemize}

\vspace{0.25in}

\begin{itemize}
\itemsep1pt\parskip0pt\parsep0pt
\item
  Other types of subsystems
\end{itemize}

\vspace{0.25in}

\begin{itemize}
\itemsep1pt\parskip0pt\parsep0pt
\item
  Assumptions about elected officials and bureaucrats being policy
  motivated
\end{itemize}

\end{frame}

\end{document}
